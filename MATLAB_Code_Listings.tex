
% ------------------------------------------------------------------------------
% REFERENCES: 
% (1)  LaTeX Source Code Listings:
% en.wikibooks.org/wiki/LaTeX/Source_Code_Listings
% (2)  MATLAB OpenSurf by Dirk-Jan Kroon:
% mathworks.com/matlabcentral/fileexchange/28300-opensurf-including-image-warp
% ------------------------------------------------------------------------------


% ------------------------------------------------------------------------------
% Copy the following before \begin{document} and edit however you like.
% ------------------------------------------------------------------------------

\lstset{ 
  language = Matlab,                  % language of the code
  basicstyle = \scriptsize,           % fontsize for the code
  keywordstyle = \color{Blue},        % \bfseries\underbar,
  commentstyle = \color{ForestGreen}, 
  stringstyle = \ttfamily,
  numbers = left,                     % placement of line-numbers
  numberstyle = \scriptsize,          % fontsize of line-numbers
  stepnumber = 1,                     % step between two line-numbers
  numbersep = 5pt,                    % spacing between line-numbers and code
  backgroundcolor = \color{White},    % background color
  showspaces = false,                 % show spaces using underscores
  showstringspaces = false,           % underline spaces within strings
  showtabs = false,                   % show tabs in strings using underscores
  frame = single,                     % adds a frame around the code
  tabsize = 2,                        % sets default tabsize to 2 spaces
  captionpos = t,                     % sets the caption-position to bottom
  breaklines = true,                  % sets automatic line breaking
  breakatwhitespace = false           % sets automatic breaks at whitespace

% title = \lstname,                   % show filenames with \lstinputlisting
                                      % also try caption instead of title
% escapeinside = {\%*}{*)},           % add comments within code
% morekeywords = {function,}          % add more keywords to the set
}        
  

% ------------------------------------------------------------------------------
% Create Source Code Listings within Document.
% i.e. between \begin{document} and \end{document}
% ------------------------------------------------------------------------------

\begin{itemize}
    \item \texttt{OpenSURF} -- This MATLAB function detects interest points in an image, and describes the points via \textit{SURF} descriptors. The inputs and outputs of the function are summarised in Listing \ref{list:opensurf}.
    \item \texttt{PaintSURF} -- This MATLAB function displays the image with the interest points (found by \texttt{OpenSURF}) drawn as circles. Circles of different sizes, directional lines, and colours are respectively associated with different scales, orientations and whether the point is a local maxima or minima. The inputs and outputs of the function are summarised in Listing \ref{list:paintsurf}.\\
\end{itemize}

% ------------------------------------------------------------------------------

\scriptsize
\begin{lstlisting} 
[caption={Inputs and Outputs of MATLAB function \texttt{OpenSURF}},label={list:opensurf}]

function ipts = OpenSURF(I,Options)

% Inputs:
%   I       - The 2D input image color or greyscale
%   Options - Optional; A struct with options (see below)
%
% Outputs:
%   Ipts - A structure with the information about all detected Landmark points
%        - Ipts.x , ipts.y  : landmark position
%        - Ipts.scale       : scale of the detected landmark
%        - Ipts.laplacian   : laplacian of the landmark neighbourhood
%        - Ipts.orientation : orientation in radians
%        - Ipts.descriptor  : descriptor for corresponding point matching
%
% Options:
%   Options.verbose     - boolean; display useful information (default False)
%   Options.upright     - boolean; determine non-rotation invariant results
%   Options.extended    - boolean; add extra landmark information to descriptor
%   Options.tresh       - Hessian response threshold (default 0.0002)
%   Options.octaves     - number of octaves to analyse (default 5)
%   Options.init_sample - initial sampling step in the image (default 2)
\end{lstlisting}
\normalsize

% ------------------------------------------------------------------------------

\scriptsize
\begin{lstlisting} 
[caption={Inputs and Outputs of MATLAB function \texttt{PaintSURF}},label={list:paintsurf}]

function PaintSURF(I,Ipts)

%  Inputs:
%    I    - image (2D color or greyscale)
%    Ipts - interest points
%
%  Outputs:
%    image I displayed with the found interest points Ipts
\end{lstlisting}
\normalsize

% ------------------------------------------------------------------------------
